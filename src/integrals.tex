\documentclass{parchment}
\usepackage{xcolor}
\author{Imię Nazwisko}
\title{Tytuł książki}

\begin{document}

% strona pierwsza

\thispagestyle{empty}
{\noindent\fontsize{18pt}{18pt}\selectfont Księgozbiór matemagiczny, tom ??}

\noindent\makebox[\linewidth]{\rule{\textwidth}{1pt}}

\newpage

% koniec strony pierwszej



% strona druga

\thispagestyle{empty}
\phantom{nothing}
\newpage

% koniec strony drugiej



% strona trzecia

\thispagestyle{empty}
{\noindent\fontsize{18pt}{18pt}\selectfont Imię Nazwisko}

\noindent\makebox[\linewidth]{\rule{\textwidth}{1pt}}

\vspace{10mm}

{\noindent\fontsize{24pt}{24pt}\selectfont \textbf{Tytuł\\(takie tam)}}
\vspace{10mm}

{\noindent\fontsize{14pt}{14pt}\selectfont Wydanie pierwsze}

\newpage

% koniec strony trzeciej



% strona czwarta

\thispagestyle{empty}
\begin{figure}[H]
\begin{minipage}[b]{.48\linewidth}
{\noindent Prof. Imię Nazwisko\\
Gdzie\\
Gdzie dalej\\
Gdzie kraj}
\end{minipage}
\end{figure}

{\noindent \textbf{Tytuł oryginału}\\Tytuł oryginału}
\vspace{5mm}

{\noindent \textbf{Okładkę zaprojektował}\\Okładka}
\vspace{5mm}

{\noindent \textbf{Zredagował}\\Redakcja}
\vspace{5mm}

{\noindent \textbf{Zredagowała technicznie}\\Redakcja techniczna}
\vspace{5mm}

{\noindent \textbf{Złożyli i połamali}\\Skład, łamanie}
\vspace{5mm}

{\noindent \textbf{Korekty dokonali}\\Korekta}

\vfill

{\noindent Copyleft by Antykwariat Czarnoksięski, Gorzów Wielkopolski 2024.
Książka, a także każda jej część, mogą być przedrukowywane oraz w jakikolwiek inny sposób reprodukowane czy powielane mechanicznie, fotooptycznie, zapisywane elektronicznie lub magnetycznie, oraz odczytywane w środkach publicznego przekazu bez pisemnej zgody wydawcy.}

\vspace{5mm}

{\noindent Przygotowano w systemie \TeX, wydrukowano na siarczystym papierze.}

% koniec strony czwartej



% strona piąta

\chapter*{Przedmowa}
Przedmowa

% koniec strony piątej



\newcommand{\N}{\mathbb N}
\newcommand{\R}{\mathbb R}

\tableofcontents

\chapter{Pochodne}


\begin{proposition}
    Pochodna jest operatorem liniowym:
    \begin{equation}
        \frac{\mathrm{d}}{\mathrm{d}x} [a f(x) + b g(x)] = a \frac{\mathrm{d}}{\mathrm{d}x} [f(x)] + b \frac{\mathrm{d}}{\mathrm{d}x} [g(x)]
    \end{equation}
\end{proposition}

\begin{proposition}[reguła Leibniza]
    \begin{equation}
        \frac{\mathrm{d}}{\mathrm{d}x} [f(x)g(x)] =  g(x) \frac{\mathrm{d}}{\mathrm{d}x} [f(x)] + f(x)\frac{\mathrm{d}}{\mathrm{d}x} [g(x)]
    \end{equation}
\end{proposition}

\begin{proof}
    Dla oszczędności miejsca, $x_h := x + h$.
\begin{align}
    \frac{\mathrm{d}}{\mathrm{d}x} [f(x)g(x)]
    & = \lim_{h \to 0} \frac{f(x_h)g(x_h) - f(x)g(x)}{h} \\
    & = \lim_{h \to 0} \frac{f(x_h)g(x_h) - f(x)g(x_h) + f(x)g(x_h) - f(x)g(x)}{h} \\
    & = \lim_{h \to 0} \frac{[f(x_h) - f(x)]g(x_h) + f(x)[g(x_h) - g(x)]}{h} \\
    & = \lim_{h \to 0} \frac{f(x_h) - f(x)}{h} \lim_{h\to 0} g(x_h) + 
        \lim_{h \to 0} \frac{g(x_h) - g(x)}{h} \lim_{h \to 0} f(x) \\
    & = g(x) \frac{\mathrm{d}}{\mathrm{d}x} [f(x)] + f(x)\frac{\mathrm{d}}{\mathrm{d}x} [g(x)],
\end{align}
    ponieważ funkcje różniczkowalne są też ciągłe.
\end{proof}

\begin{proposition}
    \label{prp:derivative_power}%
    \begin{equation}
        \frac{\mathrm{d}}{\mathrm{d}x} x^n = nx^{n-1}.
    \end{equation}
\end{proposition}

\begin{proof}
\begin{align}
    \frac{\mathrm{d}}{\mathrm{d}x} x^n
    & = \lim_{h \to 0} \frac{(x+h)^n - x^n}{h} \\
    & = \lim_{h \to 0} \frac{1}{h} \left(\sum_{k=0}^n {n \choose k} x^k h^{n-k} - x^n \right) \\
    & = \lim_{h \to 0} \frac{1}{h} \left(nx^{n-1}h + \sum_{k=0}^{n-2} {n \choose k} x^k h^{n-k}\right) \\
    & = nx^{n-1} + \lim_{h \to 0} \left(\sum_{k=0}^{n-2} {n \choose k} x^k h^{n-k-1}\right) \\
    & = nx^{n-1}.
\end{align}
\end{proof}

\chapter{Całki}

%

\section{Całkowanie przez zgadnięcie pochodnej}

Czasami wystarczy zgadnąć wynik (albo znaleźć go w tablicy pochodnych) i sprawdzić, że pasuje przez zróżniczkowanie.

\begin{integral}
    Niech $n \neq -1$.
    Wtedy
    \begin{equation}
        \int x^n \,\mathrm{d}x = \frac{x^{n+1}}{n+1}.
    \end{equation}
\end{integral}

\begin{proof}
    Wprost z \ref{prp:derivative_power}.
\end{proof}

%
\section{Całkowanie przez podstawianie}

% Banaś, Wędrychowicz, 12.18.
\begin{integral}
    $\int (\arcsin x)^2 \,\mathrm{d}x$.
\end{integral}

\begin{solution}
    Podstawiamy $u = \arcsin x$ i dostajemy całkę z $u^2 \cos u$, którą rozwiązujemy przez części, tak jak w przykładzie \ref{banas_12_14}.
\end{solution}

% Banaś, Wędrychowicz, 12.19.
\begin{integral}
    $\int \sin(\log x) \, \mathrm{d}x$.
\end{integral}

\begin{solution}
    Podstawiamy $u = \log x$, $\mathrm{d} u = \mathrm{d} x / x$, $x = \exp u$ i dostajemy całkę z $e^u \sin u$, którą rozwiązujemy przez części, tak jak w przykładzie \ref{banas_12_19_auxilia}.
\end{solution}

% Banaś, Wędrychowicz, 12.20.
\begin{integral}
    $\int \cos(\log x) \, \mathrm{d}x$.
\end{integral}    

% Banaś, Wędrychowicz: 12.29
\begin{integral}
    Banaś-Wędrychowicz, 12.29.
\end{integral}

% Banaś, Wędrychowicz: 12.30
\begin{integral}
    Banaś-Wędrychowicz, 12.30.
\end{integral}

% Banaś, Wędrychowicz: 12.31
\begin{integral}
    Banaś-Wędrychowicz, 12.31.
\end{integral}

% Banaś, Wędrychowicz: 12.32
\begin{integral}
    Banaś-Wędrychowicz, 12.32.
\end{integral}

% Banaś, Wędrychowicz: 12.33
\begin{integral}
    Banaś-Wędrychowicz, 12.33.
\end{integral}

% Banaś, Wędrychowicz: 12.34
\begin{integral}
    Banaś-Wędrychowicz, 12.34.
\end{integral}

% Banaś, Wędrychowicz: 12.35
\begin{integral}
    Banaś-Wędrychowicz, 12.35.
\end{integral}

% Banaś, Wędrychowicz: 12.36
\begin{integral}
    Banaś-Wędrychowicz, 12.36.
\end{integral}

% Banaś, Wędrychowicz: 12.37
\begin{integral}
    Banaś-Wędrychowicz, 12.37.
\end{integral}

% Banaś, Wędrychowicz: 12.38
\begin{integral}
    Banaś-Wędrychowicz, 12.38.
\end{integral}

% Banaś, Wędrychowicz: 12.39
\begin{integral}
    Banaś-Wędrychowicz, 12.39.
\end{integral}

% Banaś, Wędrychowicz: 12.40
\begin{integral}
    Banaś-Wędrychowicz, 12.40.
\end{integral}

% Banaś, Wędrychowicz: 12.41
\begin{integral}
    Banaś-Wędrychowicz, 12.41.
\end{integral}

% Banaś, Wędrychowicz: 12.42
\begin{integral}
    Banaś-Wędrychowicz, 12.42.
\end{integral}

% Banaś, Wędrychowicz: 12.43
\begin{integral}
    Banaś-Wędrychowicz, 12.43.
\end{integral}

% Banaś, Wędrychowicz: 12.44
\begin{integral}
    Banaś-Wędrychowicz, 12.44.
\end{integral}

% Banaś, Wędrychowicz: 12.45
\begin{integral}
    Banaś-Wędrychowicz, 12.45.
\end{integral}

% Banaś, Wędrychowicz: 12.46
\begin{integral}
    Banaś-Wędrychowicz, 12.46.
\end{integral}

% Banaś, Wędrychowicz: 12.47
\begin{integral}
    Banaś-Wędrychowicz, 12.47.
\end{integral}

% Banaś, Wędrychowicz: 12.48
\begin{integral}
    Banaś-Wędrychowicz, 12.48.
\end{integral}

% Banaś, Wędrychowicz: 12.49
\begin{integral}
    Banaś-Wędrychowicz, 12.49.
\end{integral}

% Banaś, Wędrychowicz: 12.50
\begin{integral}
    Banaś-Wędrychowicz, 12.50.
\end{integral}

% Banaś, Wędrychowicz: 12.51
\begin{integral}
    Banaś-Wędrychowicz, 12.51.
\end{integral}

% Banaś, Wędrychowicz: 12.52
\begin{integral}
    Banaś-Wędrychowicz, 12.52.
\end{integral}

% Banaś, Wędrychowicz: 12.53
\begin{integral}
    Banaś-Wędrychowicz, 12.53.
\end{integral}

% Banaś, Wędrychowicz: 12.54
\begin{integral}
    Banaś-Wędrychowicz, 12.54.
\end{integral}

% Banaś, Wędrychowicz: 12.55
\begin{integral}
    Banaś-Wędrychowicz, 12.55.
\end{integral}

% Banaś, Wędrychowicz: 12.56
\begin{integral}
    Banaś-Wędrychowicz, 12.56.
\end{integral}

% Banaś, Wędrychowicz: 12.57
\begin{integral}
    Banaś-Wędrychowicz, 12.57.
\end{integral}

% Banaś, Wędrychowicz: 12.58
\begin{integral}
    Banaś-Wędrychowicz, 12.58.
\end{integral}

\subsection{Podstawienia Eulera}

TODO: Banaś Wędrychowicz, 12.71 - 12.87

\begin{integral}
    Banaś-Wędrychowicz, 12.58.
\end{integral}

% https://math.stackexchange.com/questions/541751/how-prove-this-i-int-0-infty-frac1x-ln-left-frac1x1-x-right2/541861#541861
\begin{integral}[pytanie 541751 na math.stackexchange.com]
    \begin{equation}
        I = \int_0^\infty \frac{1}{x} \log \left(\frac{1+x}{1-x}\right)^2 \,\mathrm{d}x = \pi^2
    \end{equation}
\end{integral}

\begin{proof}
    Podstawiamy $y = (1+x) / (1-y)$:
    \begin{align}
        I & = 2 \int_{-1}^1 \frac{\log y^2}{1-y^2} \,\mathrm{d}y \\
          & = 8 \int_0^1 \frac{\log y}{1-y^2} \, \mathrm{d}{y} \\
          & = 8 \sum_{k=0}^\infty \int_0^1 y^{2k} \log y \,\mathrm{d} y \\
          & = 8 \sum_{k=0}^\infty \frac{1}{(2k+1)^2} \\
          & = 8 \cdot \frac{\pi^2}{8} = 8.
    \end{align}
\end{proof}
%

\section{Całkowanie przez części}

\begin{proposition}[wzór na całkowanie przez części]
\label{prp_int_by_parts}%
    Jeśli funkcje $f, g \colon I \to \R$ są różniczkowalne, to
    \begin{equation}
        \int f(x) g'(x) \,\mathrm{d}x = f(x) g(x) - \int f'(x) g(x) \,\mathrm{d} x.
    \end{equation}
\end{proposition}

\begin{proof}
    Całkujemy obie strony wzoru na pochodną iloczynu $(fg)' = fg' + f'g$, a następnie porządkujemy strony równości.
\end{proof}

% Banaś, Wędrychowicz: 12.1
\begin{integral}
    $\int x \sin x \,\mathrm{d} x$.
\end{integral}

\begin{solution}
    Całkujemy przez części, $f(x) = x$, $g'(x) = \sin x$.
    \begin{align}
        \int x \sin x \,\mathrm{d} x & = -x \cos x - \int - \cos x \, \mathrm{d}x \\
                                     & = -x \cos x + \sin x.
    \end{align}
\end{solution}

Analogicznie obliczamy całki:

% Banaś, Wędrychowicz: 12.2
\begin{integralsolved}
    $\int x \cos x \,\mathrm{d} x = x \sin x + \cos x$.
\end{integralsolved}

% Banaś, Wędrychowicz: 12.3
% Banaś, Wędrychowicz: 12.4 - podobna
% Banaś, Wędrychowicz: 12.5 - podobna
\begin{integralsolved}
    $\int x \exp x \,\mathrm{d} x = (x-1) e^x$.
\end{integralsolved}

% Banaś, Wędrychowicz: 12.6
\begin{integral}
    $\int x \arctan x \,\mathrm{d} x$.
\end{integral}

\begin{solution}
    Całkujemy przez części, $f(x) = \arctan x$, $g'(x) = x$.
    \begin{align}
        \int x \arctan x \, \mathrm{d} x & = \frac 12 x^2 \arctan x - \int \frac{x^2 \,\mathrm{d}x}{2(x^2+1)} \\
                                         & = \frac 12 x^2 \arctan x - \frac 12 \left(\int 1 \,\mathrm{d}x - \int \frac{\mathrm{d}x}{x^2+1} \right) \\
                                         & = \frac 12 x^2 \arctan x - \frac 12 \left(x - \arctan x \right) \\
                                         & = \frac 12 \left((x^2+1)\arctan x - x \right).
    \end{align}
\end{solution}

% Banaś, Wędrychowicz: 12.7
\begin{integral}
    $\int x^n \log x \,\mathrm{d} x$, gdzie $n \in \N$.
\end{integral}

\begin{solution}
    Całkujemy przez części, $f(x) = \log x$, $g'(x) = x^n$.
    \begin{align}
        \int x^n \log x \, \mathrm{d} x & = \frac{x^{n+1} \log x}{n+1} - \int \frac{x^n \,\mathrm{d} x}{n+1} \\
                                        & = \frac{x^{n+1} \log x}{n+1} - \frac{x^{n+1}}{(n+1)^2}.
    \end{align}
\end{solution}

% Banaś, Wędrychowicz: 12.8
\begin{integral}
    $\int \arccos x \,\mathrm{d} x$.
\end{integral}

\begin{solution}
    Całkujemy najpierw przez części, $f(x) = \arccos x$, $g'(x) = 1$, żeby następnie podstawić $u = 1 - x^2$, $\mathrm{d} u = -2x \mathrm{d}x$:
    \begin{align}
        \int \arccos x \, \mathrm{d} x & = x \arccos x - \int  \frac{-x \,\mathrm{d}x}{\sqrt{1-x^2}} \\
        & = x \arccos x - \frac 12 \int \frac {\mathrm{d}u}{\sqrt{u}} \\
        & = x \arccos x - \sqrt{1 - x^2}.
    \end{align}
\end{solution}

% Banaś, Wędrychowicz: 12.9
\begin{integralsolved}
    $\int \arcsin x \,\mathrm{d} x = x \arcsin x + \sqrt{1-x^2}$.
\end{integralsolved}

% Banaś, Wędrychowicz: 12.10
\begin{integral}
    $\int x (\tan x)^2 \,\mathrm{d} x$.
\end{integral}

\begin{solution}
    Całkujemy przez części, $f(x) = x$, $g'(x) = (\tan x)^2$.
    \begin{align}
        \int x (\tan x)^2 x \, \mathrm{d} x & = x (\tan x - x) - \int (\tan x - x) \,\mathrm{d}x \\
        & = x (\tan x - x) - \left(-\log(\cos(x)) - \frac{x^2}{2}\right).
    \end{align}
\end{solution}

% Banaś, Wędrychowicz: 12.11
\begin{integral}
    $\int x (\cos x)^2 \,\mathrm{d} x$.
\end{integral}

\begin{solution}
    Ponieważ $\cos 2x = 2 \cos^2 x - 1$, potrzebujemy znaleźć prostszą całkę 
    \begin{align}
        \int x \cos 2x \, \mathrm{d} x.
    \end{align}
    Całkujemy przez części: $f(x) = x$, $g'(x) = \cos 2x$, co prowadzi do jeszce prostszej całki funkcji $\sin 2x$.
    Ostatecznie
    \begin{align}
        \int x \cos 2x \, \mathrm{d} x = \frac 1 8 \left(2x^2 + 2x \sin 2x + \cos 2x\right).
    \end{align}
\end{solution}

% Banaś, Wędrychowicz, 12.12 to całka z x log(x^2+1), ale tam wystarczy podstawić u = x^2 + 1, wtedy du = 2x dx.
% Banaś, Wędrychowicz, 12.16
% Banaś, Wędrychowicz, 12.17.
\begin{integral}
    $\int (\log x)^n \,\mathrm{d}x$.
\end{integral}

\begin{solution}
    Całkujemy przez części, $f(x) = (\log x)^n$, $g'(x) = 1$.
    Dostajemy początek rekurencji:
    \begin{equation}
        \int (\log x)^n \, \mathrm{d}x = x (\log x)^n - n \int (\log x)^{n-1} \,\mathrm{d} x.
    \end{equation}
    z warunkiem brzegowym:
    \begin{equation}
        \int \log x\, \mathrm{d}x = x\log x - x.
    \end{equation}
\end{solution}

% Banaś, Wędrychowicz, 12.13.
\begin{integral}
    Niech $n$ będzie liczbą naturalną, wtedy
    \begin{equation}
        I_n = \int x^n e^x \,\mathrm{d} x = e^x \sum_{k=0}^n (-1)^{n-k} \frac{n!}{k!}x^k.
    \end{equation}
\end{integral}

\begin{solution}
    Dowiedziemy tego indukcyjnie.
    Dla $n = 0$, całka jest elementarna.
    Jeżeli $n \ge 1$, to całkujemy przez części: $f(x) = x^n$, $g'(x) = e^x$ i dostajemy zależność rekurencyjną
    \begin{equation}
        I_n = x^n e^x - nI_{n-1}.
    \end{equation}
\end{solution}

% Banaś, Wędrychowicz, 12.14.
% Banaś, Wędrychowicz, 12.15.
\begin{integral}
    \label{banas_12_14}%
    $\int x^3 \sin x \, \mathrm{d}x$.
\end{integral}

\begin{solution}
    Całkujemy przez części, $f(x) = x^3$, $g'(x) = \sin x$.
    Dostajemy początek rekurencji:
    \begin{equation}
        \int x^3 \sin x \, \mathrm{d}x = - x^3 \cos x - \int - 3x^2 \cos x \,\mathrm{d}x
    \end{equation}
    rozwiązaniem której jest $3 (x^2-2) \sin x + x (6-x^2) \cos x$.
\end{solution}

\begin{integral}
    % pomocnicza dla Banaś 12.19
    \label{banas_12_19_auxilia}%
    \begin{equation}
        \int e^x \sin x \,\mathrm{d}x = \frac {e^x} 2 (\sin x - \cos x).
    \end{equation}
\end{integral}

% Banaś, Wędrychowicz, 12.21.
\begin{integral}
Banaś-Wędrychowicz, 12.21. % \sqrt(k + x^2)
\end{integral}

% Banaś, Wędrychowicz, 12.22.
\begin{integral}
Banaś-Wędrychowicz, 12.22. % x^2 e^x sin x
\end{integral}

% Banaś, Wędrychowicz, 12.23.
\begin{integral}
Banaś-Wędrychowicz, 12.23. % x / (sin ^2 x)
\end{integral}

% Banaś, Wędrychowicz, 12.24.
\begin{integral}
Banaś-Wędrychowicz, 12.24. % x arcsin x / (1 - x^2)
\end{integral}

% Banaś, Wędrychowicz, 12.25.
\begin{integral}
Banaś-Wędrychowicz, 12.25. % x^2 cos 4x
\end{integral}

% Banaś, Wędrychowicz, 12.26.
\begin{integral}
\begin{equation}
    \int \frac{x \log(\sqrt{x^2+1}+x)}{\sqrt{x^2+1}} \,\mathrm{d}x = \sqrt{x^2+1} \arsinh x - x.
\end{equation}
\end{integral}

\begin{solution}
    Zauważamy, że $\log(\sqrt{x^2+1} + x) = \arsinh x$, a następnie całkujemy przez części: $f(x) = \arsinh x$, $g'(x) = x / \sqrt{x^2+1}$, wtedy $f(x) = 1/\sqrt{x^2+1}$, $g(x) = \sqrt{x^2+1}$.
\end{solution}

% Banaś, Wędrychowicz, 12.27.
% \begin{integral}
% Banaś-Wędrychowicz, 12.27. % arc tg sqrt (x)
% \end{integral}

% Banaś, Wędrychowicz, 12.28.
% \begin{integral}
% Banaś-Wędrychowicz, 12.28. % x e ^ (arctg x) / (1+x^2)^1.5
% \end{integral}

%
\section{Całkowanie funkcji wymiernych}
Całkowanie funkcji wymiernych

\begin{integral}
\begin{equation}
    \int_0^\infty \frac{x^8-4x^6+9x^4-5x^2+1}{x^{12}-10x^{10}+37x^8-42x^6+26x^4-8x^2+1} \,\mathrm{d}x = \frac{\pi}{2}
\end{equation}
\end{integral}

\begin{solution}
    Strona 258 w: Bailey, D. H.; Borwein, J. M.; Calkin, N. J.; Girgensohn, R.; Luke, D. R.; and Moll, V. H. Experimental Mathematics in Action. Wellesley, MA: A K Peters, 2007.
\end{solution}

\section{Całkowanie różniczek dwumiennych}
Całkowanie różniczek dwumiennych

\section{Całkowanie funkcji trygonometrycznych}
Całkowanie funkcji trygonometrycznych

\section{Trudne całki}

% TODO: różniczkowanie pod znakiem całki
% https://math.stackexchange.com/questions/942263/really-advanced-techniques-of-integration-definite-or-indefinite
\begin{integral}
    $\int_0^\infty \sin(x) / x \,\mathrm{d}x = \pi/2$.
\end{integral}

% https://math.stackexchange.com/a/9292/1298830
\begin{integral}
    $\int_{-\infty}^\infty \exp(-x^2) \,\mathrm{d}x = \sqrt x$.
\end{integral}

% https://math.stackexchange.com/questions/580521/generalizing-int-01-frac-arctan-sqrtx2-2-sqrtx2-2
\begin{integral}[całka Ahmeda]
    \begin{equation}
        \int_0^1 \frac{\arctan \sqrt{x^2+2}}{(x^2+1) \sqrt{x^2+2}} \,\mathrm{d}x = \frac{5\pi^2}{96}.
    \end{equation}
\end{integral}

\begin{solution}
    Zafar Ahmed -- \emph{Definitely An Integral} w Amer. Math. Monthly 109, 670-671, 2002.
\end{solution}

% https://math.stackexchange.com/q/562694
\begin{integral}[pytanie 562694 na math.stackexchange.com]
    \begin{equation}
        \int_{-1}^1 \frac{1}{x} \sqrt{\frac{1+x}{1-x}} \log \frac{2x^2+2x+1}{2x^2-2x+1} \,\mathrm{d}x = 4 \pi \arccot \sqrt{\phi}.
    \end{equation}
\end{integral}


% https://math.stackexchange.com/questions/tagged/integration?tab=votes&page=2&pagesize=15

% https://math.stackexchange.com/questions/457231/how-to-prove-int-infty-infty-fxdx-int-infty-infty-f-left/457271#457271

Another challenging integral is that for the volume of the Reuleaux tetrahedron,
% V	=	int_0^1[(8sqrt(3))/(1+3t^2)-(16sqrt(2)(3t+1)(4t^2+t+1)^(3/2))/((3t^2+1)(11t^2+2t+3)^2)-(sqrt(2)(249t^2+54t+65))/((11t^2+2t+3)^2)]dt,	
% (35)	=	8/3pi-(27)/4cos^(-1)(1/3)+1/4sqrt(2)	(36)=	0.4221577...	 (37)  (OEIS A102888; Weisstein). 

% https://mathworld.wolfram.com/images/gifs/FoxTrotMathTest.jpg -> koniec https://mathworld.wolfram.com/DefiniteIntegral.html

% https://www.amazon.com/dp/303002461X/?coliid=I23YWXHH3UPCFX&colid=2V83EEIVCDHZL&psc=1&ref_=lv_ov_lig_dp_it



% https://www.amazon.com/dp/0867202939

\end{document}