\documentclass{createspace}
\usepackage{xcolor}
\author{Imię Nazwisko}
\title{Tytuł książki}

\begin{document}

% strona pierwsza

\thispagestyle{empty}
{\noindent\fontsize{18pt}{18pt}\selectfont Księgozbiór matemagiczny, tom ??}

\noindent\makebox[\linewidth]{\rule{\textwidth}{1pt}}

\newpage

% koniec strony pierwszej


\input{00_intro/head_2}

% strona trzecia

\thispagestyle{empty}
{\noindent\fontsize{18pt}{18pt}\selectfont Imię Nazwisko}

\noindent\makebox[\linewidth]{\rule{\textwidth}{1pt}}

\vspace{10mm}

{\noindent\fontsize{24pt}{24pt}\selectfont \textbf{Tytuł\\(takie tam)}}
\vspace{10mm}

{\noindent\fontsize{14pt}{14pt}\selectfont Wydanie pierwsze}

\newpage

% koniec strony trzeciej



% strona czwarta

\thispagestyle{empty}
\begin{figure}[H]
\begin{minipage}[b]{.48\linewidth}
{\noindent Prof. Imię Nazwisko\\
Gdzie\\
Gdzie dalej\\
Gdzie kraj}
\end{minipage}
\end{figure}

{\noindent \textbf{Tytuł oryginału}\\Tytuł oryginału}
\vspace{5mm}

{\noindent \textbf{Okładkę zaprojektował}\\Okładka}
\vspace{5mm}

{\noindent \textbf{Zredagował}\\Redakcja}
\vspace{5mm}

{\noindent \textbf{Zredagowała technicznie}\\Redakcja techniczna}
\vspace{5mm}

{\noindent \textbf{Złożyli i połamali}\\Skład, łamanie}
\vspace{5mm}

{\noindent \textbf{Korekty dokonali}\\Korekta}

\vfill

{\noindent Copyleft by Antykwariat Czarnoksięski, Gorzów Wielkopolski 2024.
Książka, a także każda jej część, mogą być przedrukowywane oraz w jakikolwiek inny sposób reprodukowane czy powielane mechanicznie, fotooptycznie, zapisywane elektronicznie lub magnetycznie, oraz odczytywane w środkach publicznego przekazu bez pisemnej zgody wydawcy.}

\vspace{5mm}

{\noindent Przygotowano w systemie \TeX, wydrukowano na siarczystym papierze.}

% koniec strony czwartej



% strona piąta

\chapter*{Przedmowa}
Przedmowa

% koniec strony piątej


\tableofcontents

\chapter{Pochodne}


\begin{proposition}
    Pochodna jest operatorem liniowym:
    \begin{equation}
        \frac{\mathrm{d}}{\mathrm{d}x} [a f(x) + b g(x)] = a \frac{\mathrm{d}}{\mathrm{d}x} [f(x)] + b \frac{\mathrm{d}}{\mathrm{d}x} [g(x)]
    \end{equation}
\end{proposition}

\begin{proposition}[reguła Leibniza]
    \begin{equation}
        \frac{\mathrm{d}}{\mathrm{d}x} [f(x)g(x)] =  g(x) \frac{\mathrm{d}}{\mathrm{d}x} [f(x)] + f(x)\frac{\mathrm{d}}{\mathrm{d}x} [g(x)]
    \end{equation}
\end{proposition}

\begin{proof}
    Dla oszczędności miejsca, $x_h := x + h$.
\begin{align}
    \frac{\mathrm{d}}{\mathrm{d}x} [f(x)g(x)]
    & = \lim_{h \to 0} \frac{f(x_h)g(x_h) - f(x)g(x)}{h} \\
    & = \lim_{h \to 0} \frac{f(x_h)g(x_h) - f(x)g(x_h) + f(x)g(x_h) - f(x)g(x)}{h} \\
    & = \lim_{h \to 0} \frac{[f(x_h) - f(x)]g(x_h) + f(x)[g(x_h) - g(x)]}{h} \\
    & = \lim_{h \to 0} \frac{f(x_h) - f(x)}{h} \lim_{h\to 0} g(x_h) + 
        \lim_{h \to 0} \frac{g(x_h) - g(x)}{h} \lim_{h \to 0} f(x) \\
    & = g(x) \frac{\mathrm{d}}{\mathrm{d}x} [f(x)] + f(x)\frac{\mathrm{d}}{\mathrm{d}x} [g(x)],
\end{align}
    ponieważ funkcje różniczkowalne są też ciągłe.
\end{proof}

\begin{proposition}
    \label{prp:derivative_power}%
    \begin{equation}
        \frac{\mathrm{d}}{\mathrm{d}x} x^n = nx^{n-1}.
    \end{equation}
\end{proposition}

\begin{proof}
\begin{align}
    \frac{\mathrm{d}}{\mathrm{d}x} x^n
    & = \lim_{h \to 0} \frac{(x+h)^n - x^n}{h} \\
    & = \lim_{h \to 0} \frac{1}{h} \left(\sum_{k=0}^n {n \choose k} x^k h^{n-k} - x^n \right) \\
    & = \lim_{h \to 0} \frac{1}{h} \left(nx^{n-1}h + \sum_{k=0}^{n-2} {n \choose k} x^k h^{n-k}\right) \\
    & = nx^{n-1} + \lim_{h \to 0} \left(\sum_{k=0}^{n-2} {n \choose k} x^k h^{n-k-1}\right) \\
    & = nx^{n-1}.
\end{align}
\end{proof}

\chapter{Całki}

\begin{integral}
    Niech $n \neq -1$.
    Wtedy
\begin{equation}
    \int x^n \,\mathrm{d}x = \frac{x^{n+1}}{n+1}.
\end{equation}
\end{integral}

\begin{proof}
    Wprost z \ref{prp:derivative_power}.
\end{proof}

\end{document}