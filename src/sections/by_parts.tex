%

\section{Całkowanie przez części}

\begin{proposition}[wzór na całkowanie przez części]
\label{prp_int_by_parts}%
    Jeśli funkcje $f, g \colon I \to \R$ są różniczkowalne, to
    \begin{equation}
        \int f(x) g'(x) \,\mathrm{d}x = f(x) g(x) - \int f'(x) g(x) \,\mathrm{d} x.
    \end{equation}
\end{proposition}

\begin{proof}
    Całkujemy obie strony wzoru na pochodną iloczynu $(fg)' = fg' + f'g$, a następnie porządkujemy strony równości.
\end{proof}

% Banaś, Wędrychowicz: 12.1
\begin{integral}
    $\int x \sin x \,\mathrm{d} x$.
\end{integral}

\begin{solution}
    Całkujemy przez części, $f(x) = x$, $g'(x) = \sin x$.
    \begin{align}
        \int x \sin x \,\mathrm{d} x & = -x \cos x - \int - \cos x \, \mathrm{d}x \\
                                     & = -x \cos x + \sin x.
    \end{align}
\end{solution}

Analogicznie obliczamy całki:

% Banaś, Wędrychowicz: 12.2
\begin{integralsolved}
    $\int x \cos x \,\mathrm{d} x = x \sin x + \cos x$.
\end{integralsolved}

% Banaś, Wędrychowicz: 12.3
\begin{integralsolved}
    $\int x \exp x \,\mathrm{d} x = (x-1) e^x$.
\end{integralsolved}

% Banaś, Wędrychowicz: 12.6
\begin{integral}
    $\int x \arctan x \,\mathrm{d} x$.
\end{integral}

\begin{solution}
    Całkujemy przez części, $f(x) = \arctan x$, $g'(x) = x$.
    \begin{align}
        \int x \arctan x \, \mathrm{d} x & = \frac 12 x^2 \arctan x - \int \frac{x^2 \,\mathrm{d}x}{2(x^2+1)} \\
                                         & = \frac 12 x^2 \arctan x - \frac 12 \left(\int 1 \,\mathrm{d}x - \int \frac{\mathrm{d}x}{x^2+1} \right) \\
                                         & = \frac 12 x^2 \arctan x - \frac 12 \left(x - \arctan x \right) \\
                                         & = \frac 12 \left((x^2+1)\arctan x - x \right).
    \end{align}
\end{solution}

% Banaś, Wędrychowicz: 12.7
\begin{integral}
    $\int x^n \log x \,\mathrm{d} x$, gdzie $n \in \N$.
\end{integral}

\begin{solution}
    Całkujemy przez części, $f(x) = \log x$, $g'(x) = x^n$.
    \begin{align}
        \int x^n \log x \, \mathrm{d} x & = \frac{x^{n+1} \log x}{n+1} - \int \frac{x^n \,\mathrm{d} x}{n+1} \\
                                        & = \frac{x^{n+1} \log x}{n+1} - \frac{x^{n+1}}{(n+1)^2}.
    \end{align}
\end{solution}

% Banaś, Wędrychowicz: 12.8
\begin{integral}
    $\int \arccos x \,\mathrm{d} x$.
\end{integral}

\begin{solution}
    Całkujemy najpierw przez części, $f(x) = \arccos x$, $g'(x) = 1$, żeby następnie podstawić $u = 1 - x^2$, $\mathrm{d} u = -2x \mathrm{d}x$:
    \begin{align}
        \int \arccos x \, \mathrm{d} x & = x \arccos x - \int  \frac{-x \,\mathrm{d}x}{\sqrt{1-x^2}} \\
        & = x \arccos x - \frac 12 \int \frac {\mathrm{d}u}{\sqrt{u}} \\
        & = x \arccos x - \sqrt{1 - x^2}.
    \end{align}
\end{solution}

% Banaś, Wędrychowicz: 12.9
\begin{integralsolved}
    $\int \arcsin x \,\mathrm{d} x = x \arcsin x + \sqrt{1-x^2}$.
\end{integralsolved}

% Banaś, Wędrychowicz: 12.10
\begin{integral}
    $\int x (\tan x)^2 \,\mathrm{d} x$.
\end{integral}

\begin{solution}
    Całkujemy przez części, $f(x) = x$, $g'(x) = (\tan x)^2$.
    \begin{align}
        \int x (\tan x)^2 x \, \mathrm{d} x & = x (\tan x - x) - \int (\tan x - x) \,\mathrm{d}x \\
        & = x (\tan x - x) - \left(-\log(\cos(x)) - \frac{x^2}{2}\right).
    \end{align}
\end{solution}

% Banaś, Wędrychowicz: 12.11
\begin{integral}
    $\int x (\cos x)^2 \,\mathrm{d} x$.
\end{integral}

\begin{solution}
    Ponieważ $\cos 2x = 2 \cos^2 x - 1$, potrzebujemy znaleźć prostszą całkę 
    \begin{align}
        \int x \cos 2x \, \mathrm{d} x.
    \end{align}
    Całkujemy przez części: $f(x) = x$, $g'(x) = \cos 2x$, co prowadzi do jeszce prostszej całki funkcji $\sin 2x$.
    Ostatecznie
    \begin{align}
        \int x \cos 2x \, \mathrm{d} x = \frac 1 8 \left(2x^2 + 2x \sin 2x + \cos 2x\right).
    \end{align}
\end{solution}

% Banaś, Wędrychowicz, 12.12 to całka z x log(x^2+1), ale tam wystarczy podstawić u = x^2 + 1, wtedy du = 2x dx.
\begin{integral}
    $\int \log(x) \,\mathrm{d}x = x\log x - x$.
\end{integral}

\begin{solution}
    Całkujemy przez części, $f = \log(x)$, $g'(x) = 1$.
\end{solution}

% Banaś, Wędrychowicz, 12.13.
\begin{integral}
    Niech $n$ będzie liczbą naturalną, wtedy
    \begin{equation}
        I_n = \int x^n e^x \,\mathrm{d} x = e^x \sum_{k=0}^n (-1)^{n-k} \frac{n!}{k!}x^k.
    \end{equation}
\end{integral}

\begin{solution}
    Dowiedziemy tego indukcyjnie.
    Dla $n = 0$, całka jest elementarna.
    Jeżeli $n \ge 1$, to całkujemy przez części: $f(x) = x^n$, $g'(x) = e^x$ i dostajemy zależność rekurencyjną
    \begin{equation}
        I_n = x^n e^x - nI_{n-1}.
    \end{equation}
\end{solution}

% Banaś, Wędrychowicz, 12.14.
\begin{integral}
Banaś-Wędrychowicz, 12.14.
\end{integral}

% Banaś, Wędrychowicz, 12.15.
\begin{integral}
Banaś-Wędrychowicz, 12.15.
\end{integral}

% Banaś, Wędrychowicz, 12.16.
\begin{integral}
Banaś-Wędrychowicz, 12.16.
\end{integral}

% Banaś, Wędrychowicz, 12.17.
\begin{integral}
Banaś-Wędrychowicz, 12.17.
\end{integral}

% Banaś, Wędrychowicz, 12.18.
\begin{integral}
Banaś-Wędrychowicz, 12.18.
\end{integral}

% Banaś, Wędrychowicz, 12.19.
\begin{integral}
Banaś-Wędrychowicz, 12.19.
\end{integral}

% Banaś, Wędrychowicz, 12.20.
\begin{integral}
Banaś-Wędrychowicz, 12.20.
\end{integral}

% Banaś, Wędrychowicz, 12.21.
\begin{integral}
Banaś-Wędrychowicz, 12.21.
\end{integral}

% Banaś, Wędrychowicz, 12.22.
\begin{integral}
Banaś-Wędrychowicz, 12.22.
\end{integral}

% Banaś, Wędrychowicz, 12.23.
\begin{integral}
Banaś-Wędrychowicz, 12.23.
\end{integral}

% Banaś, Wędrychowicz, 12.24.
\begin{integral}
Banaś-Wędrychowicz, 12.24.
\end{integral}

% Banaś, Wędrychowicz, 12.25.
\begin{integral}
Banaś-Wędrychowicz, 12.25.
\end{integral}

% Banaś, Wędrychowicz, 12.26.
\begin{integral}
Banaś-Wędrychowicz, 12.26.
\end{integral}

% Banaś, Wędrychowicz, 12.27.
\begin{integral}
Banaś-Wędrychowicz, 12.27.
\end{integral}

% Banaś, Wędrychowicz, 12.28.
\begin{integral}
Banaś-Wędrychowicz, 12.28.
\end{integral}

%